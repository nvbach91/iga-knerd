\begin{abstract}


We carried out a literature survey on ontologies %involving 
dealing with the scholarly and research domains, %to create a comprehensive overview of available knowledge-based resources on this topic. 
with focus on modeling the knowledge graphs that would support information foraging by researchers within the different roles they fulfill during their career.
In the state of the art we identified 43 relevant ontologies, of which 34 were found sufficiently documented to be reusable. 
%that were developed specifically for use cases related to the representation of these domains. 
At the same time, based on the analysis of extensive CVs and activity logs of two senior researchers%and also taking into account some existing scholarly knowledge graphs
, we formulated a structured set of competency questions %about research-related 
that could be answered through information foraging on the web, and created a high-level conceptual model indicating the data structures that would provide answers to these questions in a holistic knowledge graph. We then studied the retrieved ontologies and mapped them on the entities and relationships from our conceptual model. We identified many overlaps between the ontologies, as well as a few missing features. Preliminary proposals for dealing with some of the overlaps and gaps were formulated.





% The scholarly/academic domain is vast and heterogeneous. As such, it presents an interesting challenge to the semantic web technologies...

% We find ontologies that can together cover our case study - everything related to researcher life

\keywords{
%academic ontologies \and bibliographic ontologies
scholarly ontology \and literature survey
\and competency question \and knowledge graph \and %ontology analysis \and %research domain \and
%research-related
information foraging %\and semantic web.
}
\end{abstract}