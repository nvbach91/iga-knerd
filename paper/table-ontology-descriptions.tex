

\medskip
\noindent \textbf{Description of ontologies}

\medskip
\noindent \textbf{OLOUD-BASE} \& \textbf{OLOUD-LOC}. \cite{10.12700/APH.14.4.2017.4.4} The objectives of the OLOUD ontology is to support the development and publishing of Linked Open University Datasets and the applications built on the top of these Open Datasets. OLOUD contains classes and properties to describe Organizations, People, their Roles and Publications, Subjects, Courses and other Events and their temporal and spatial description. The ontology is divided into two connected modules: 1. OLOUD-BASE is the main module describing all the university related concepts and uses the prefix \texttt{oloud}, 2. OLOUD-LOC module provides the indoor location and navigation features and uses the prefix \texttt{loc}.

\medskip
\noindent \textbf{VIVO}. \cite{DBLP:series/synthesis/2012Borner} An ontology of academic and research domain, developed in the framework of the VIVO project. It represents researchers in the context of their experience, outputs, interests, accomplishments, and associated institutions.

\medskip
\noindent \textbf{SWRC}. \cite{DBLP:conf/epia/SureBHHO05} An ontology for modeling entities of research communities such as persons, organisations, publications (bibliographic metadata) and their relationship.

\medskip
\noindent \textbf{ABET}. \cite{10.14569/IJACSA.2016.070717} Ontology of Accreditation Board of Engineering and Technology Process. It helps faculty  or  curriculum  committees  avoid  over  mapping  or  under mapping students' outcomes.

\medskip
\noindent \textbf{CCSO}. \cite{DBLP:conf/esws/KatisKAV18} This Ontology aims to provide data model for describing the subjects of Curriculum, Course and Syllabus in Higher education. Using this ontology, syllabus items can be effectively described and annotated enabling intelligent systems to support teaching and learning by offering automated services like syllabus semantic searching, matching and interlinking, syllabus recommendation and evolution.

\medskip
\noindent \textbf{FOAF-Academic}. \cite{DBLP:conf/incos/KalemiM11} A Vocabulary for the Academic Community. This ontology helps the academic community in saying anything about their achievements, their qualifications, activities and the communities that are near to them.

\medskip
\noindent \textbf{AIISO}. \cite{DBLP:conf/incos/KalemiM11} The Academic Institution Internal Structure Ontology (AIISO) provides classes and properties to describe the internal organizational structure of an academic institution.

\medskip
\noindent \textbf{ESO} \& \textbf{EAO}. \cite{DBLP:conf/semweb/RashidM18} Education Application Ontology and Education Application Ontology are used to simplify lesson planning for teachers, providing support for students by linking relevant resources, and providing a potential terminology for use in a lingua-franca for communicating with multiple communities about education components.

\medskip
\noindent \textbf{PLET4Thesis}. \cite{DBLP:conf/icetc/Tapia-LeonSCL17} The PLET4Thesis ontology is designed in order to organise the process of thesis development using the elements required to create a PLE (personal leaning environment).  Designed to guide thesis students in the construction of their PLE.

\medskip
\noindent \textbf{ORKG}. \cite{DBLP:conf/kcap/OelenJFSA19} The Open Research Knowledge Graph Ontology is used for comparing research contributions in a scholarly knowledge graph.

\medskip
\noindent \textbf{Researcher Profile Ontology}. \cite{DBLP:conf/cvc/BravoRC19} This ontology is designed specifically to represent academic contexts at a public university. It supports the representation of researcher profiles in a given academic environment.

\medskip
\noindent \textbf{IRAO}. Used to describe research artifacts in research, including authorship, evolution, etc.

\medskip
\noindent \textbf{CERIF}. \cite{DBLP:journals/procedia/JorgLK12} The Common European Research Information Format (CERIF) Ontology Specification provides basic concepts and properties for describing research information as semantic data. This document contains a friendly description of the Common European Research Information Format (CERIF) Ontology developed by EuroCRIS.

\medskip
\noindent \textbf{RO}. \cite{DBLP:journals/ws/Belhajjame0GGHP15} The Research Object Ontology provides basic structure for the description of aggregated resources and the annotations that are made on those resources.




%add by gollam rabby
% start

\medskip
\noindent \textbf{FaBiO}. FaBiO\cite{DBLP:journals/ws/PeroniS12} is an ontology applied for describing the entities that are published or undeniably publishable works (e.g. journal articles, conference papers, books), that contain or suggest bibliographic references. It also covers data sets, web pages, blogs, computer algorithms, formal specifications and vocabularies, experimental protocols, legal records, technical and commercial reports, governmental papers and similar publications, and also anthologies, catalogs, and corresponding collections. FaBiO has been developed to overcome any restriction to the classes and the domains with ranges of its properties. It is flexible and it has a superb advantage of allowing itself to be used side by side with other models.

\medskip
\noindent \textbf{CiTO}. CiTO\cite{DBLP:journals/biomedsem/Shotton10} is an ontology that allows characterization of the character or type of citations, both factually and rhetorically. Properties and consequently their inverses could even be classified as rhetorical and/or factual, with the rhetorical properties being grouped in three sets: positive, informative (neutral), or negative. The domain and range constraints from this thing and also properties aren't defined, so this ontology is easily integrable with other ontology models, like FaBiO. Two other ontologies are defined explicitly for describing specific aspects of CiTO, the first one is called Functions of Citations Ontology (FOCO) and the other one is called CiTO to Wordnet Ontology (C2W).

\medskip
\noindent \textbf{BiRO}. BiRO\cite{DBLP:conf/esws/IorioNPSV14} is an ontology designed to define the bibliographic records, bibliographic references, and their compilation into bibliographic collections and lists. This ontology also uses an OWL-based definition of the FRBR model\cite{bowen2011frbr} to identify the bibliographic references and their compilation into ordered bibliographic lists.

\medskip
\noindent \textbf{C4O}. C4O\cite{DBLP:conf/semweb/OsbornePM14} also provides the ontological structures which permit to record the amount of in-text citations, alongside their textual citation contexts and also the number of citations a cited entity has received globally on a specific date. It is also useful to explain how references are utilized in the citing paper. 

\medskip
\noindent \textbf{DoCO}. DoCO\cite{DBLP:journals/semweb/ConstantinPPSV16} is an ontology that organizes structured vocabulary from written document components following both structural (e.g. block, inline, paragraph, section, chapter) and rhetorical (e.g. introduction, discussion, acknowledgments, reference list, figure, appendix) options. It also imports the Pattern Ontology which describes the structural patterns, and therefore the Discourse Element Ontology (DEO), which was developed with DoCO to explain the rhetorical components. DoCO also defines hybrid classes to describe elements that are both structural and rhetorical in nature, like paragraph, section, or list. In addition, it aligned with the SALT Rhetorical Ontology and therefore the Ontology of Rhetorical Blocks (ORB).


\medskip
\noindent \textbf{PSO}. PSO\cite{DBLP:conf/i-semantics/PeroniSV12} ontology is designed and developed according to the TVC pattern, to characterize the publication status (e.g. draft, submitted, under review, etc.) of documents at every stage of the publishing process. Documents hold a specific status at a specific time as an immediate consequence of a particular event. Other pre-existing ontologies describing the status of documents (e.g. BIBO), rely on the specific property links but PSO prevents a correct description of scenarios.  

\medskip
\noindent \textbf{PRO}. PRO\cite{DBLP:conf/i-semantics/PeroniSV12} is an ontology that also uses the TVC pattern for characterization of the roles of agents (e.g. people, corporate bodies, and computational agents) within the publication process. Most of the time, the agents are authors, editors, reviewers, publishers, or librarians. It defines publishing roles as crucial as providing an entire description of a scholarly resource like a paper or a dataset.  

\medskip
\noindent \textbf{PWO}. PWO\cite{DBLP:journals/semweb/GangemiPSV17} is an ontology with two main classes called Workflow and Step, used for describing the steps of the workflow related to the publication of a document or other publication entity.

\medskip
\noindent \textbf{SCoRO}.SCoRO\cite{Scoro} is the Scholarly Contributions and Roles Ontology (CERIF-compliant ontology) for describing the contributions of authors, publishers, students, and research administrators. It is possibly used in those organizations where they're members with concerning projects, research investigations, and other academic activities focusing on scholarly journal articles and other outputs.

\medskip
\noindent \textbf{FRAPO} FRAPO\cite{Frapo} is the Funding Research Administration and Projects Ontology, which also uses a CERIF-compliant ontology for describing administrative information concerning grant funding and research projects. It also imports FOAF for characterizing people. It is mostly used for the characterization of grant applications, funding bodies, research projects, project partners, etc (the type of data stored in Current Research Information Systems (CRIS)). It also can be described as other sorts of projects, for instance, building projects and academic projects.

\medskip
\noindent \textbf{DataCite}. The main intent of the DataCite\cite{DataCite_Ontology} Ontology is to stock a versatile mechanism to prescribe the identifier for bibliographic resources and related entities. it also allows supplying a link between a resource and therefore the document describing its metadata utilizing CiTO, using the property citesAsMetadataDocument, and FaBiO, through the category of MetadataDocument. Additionally, to those entities, the DataCite Ontology provides appropriate classes and properties to specify the actual scheme followed to create the resource metadata exemplified within the metadata document.


\medskip
\noindent \textbf{BiDO}. BiDO\cite{tapia2019extension} developed a well-known model for enabling the classification of authors and journals consistent with bibliometric data, also share and reuse the data during a different context. The core module of the ontology allows one to explain any entity and therefore the related bibliometric data at a particular time and consistent with a particular agent. BiDO consists of three different modules and people are Standard bibliometric measures, Research career categories, and Review measures.

\medskip
\noindent \textbf{FiveStars}. To measure the standard of any multidimensional online journal a five stars' constellation like approach is followed by Five Stars of Online Journal Articles. The properties which are considered as stars are review quality, accessibility, content enrichment, availability of datasets and the machine-readable metadata. Articles of the online journals are evaluated based on these properties to enhance research quality and communications. This five star based conceptual framework is undoubtedly very helpful for researchers and other personnel's associated with journal publishing.\cite{DBLP:journals/dlib/Shotton12}

\medskip
\noindent \textbf{FAIR}.  FAIR \cite{Fair} is a review ontology (FR) that identifies a group of classes, properties, and axioms for describing research reviews as semantic objects and also reuse standard existing vocabularies by utilizing ontology engineering techniques.


\medskip
\noindent \textbf{BSBM}. BSBM\cite{tapia2019extension} is an enhancement of the BiDO Ontology, particularly BiDO Standard Bibliometric Measures.


\medskip
\noindent \textbf{OC}. The OpenCitations Data Model (OCDM)\cite{DBLP:journals/corr/abs-1906-11964} is the metadata model used for the stored information all together in the OpenCitation datasets. OCDM also allows us to record information about published bibliographic resources,https://www.overleaf.com/project/5ebbc2be39c6e30001174487 possible resource embodiments, bibliographic references, responsible agents, roles, citations, and external identifiers.


\medskip

 %end